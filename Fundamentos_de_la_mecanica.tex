\documentclass[a4paper,10pt]{article}
\usepackage[utf8]{inputenc}
\usepackage[spanish,es-lcroman]{babel}
\usepackage[affil-it]{authblk}
\usepackage{enumerate}
\usepackage{graphicx}
\usepackage{hyperref}
\usepackage{amsmath}
\usepackage{amssymb}
\usepackage{cancel}
\usepackage[usenames, dvipsnames]{color}
\usepackage{tikz}
\usepackage[labelfont=bf]{caption}
\usepackage{subcaption} %Multiple images
\usepackage{multicol} % Multiple columns
\usepackage{float}
\usepackage{cleveref}
 \usepackage{relsize} % bigger math symbols
\usepackage[margin=1.4in]{geometry}
\usepackage[titletoc,toc,title]{appendix}
\usepackage{enumitem}
\usepackage{etoolbox}
\usetikzlibrary{calc}
\numberwithin{equation}{section}

\graphicspath{Tarea10/}

% Circled words
\newcommand{\circled}[2][]{%
  \tikz[baseline=(char.base)]{%
    \node[shape = circle, draw, inner sep = 1pt]
    (char) {\phantom{\ifblank{#1}{#2}{#1}}};%
    \node at (char.center) {\makebox[0pt][c]{#2}};}}
\robustify{\circled}

%Appendices in spanish
\renewcommand{\appendixname}{Ap\'endices}
\renewcommand{\appendixtocname}{Ap\'endices}
\renewcommand{\appendixpagename}{Ap\'endices}

%Zero delimiter
\newcommand{\zerodel}{.\kern-\nulldelimiterspace}

%Columns separation
\setlength{\columnsep}{1cm}

%Indentation
\setlength{\parindent}{0ex}

%Multiple References

\crefrangelabelformat{equation}{(#3#1#4--#5\crefstripprefix{#1}{#2}#6)}

\usepackage{xparse}

%Boxes

\newcommand*{\boxcolor}{blue}
\makeatletter
\renewcommand{\boxed}[1]{\textcolor{\boxcolor}{%
\tikz[baseline={([yshift=-1ex]current bounding box.center)}] \node [rectangle, minimum width=1ex,rounded corners,draw] {\normalcolor\m@th$\displaystyle#1$};}}
 \makeatother

%Constantes
\newcommand{\euler}{\mathrm{e}}
\newcommand{\im}{i}

%Lemas, teoremas, definiciones y pruebas
\newcommand{\definicion}{\textbf{Definición: }}
\newcommand{\lema}{\textbf{Lema: }}
\newcommand{\teorema}{\textbf{Teorema: }}
\newcommand{\prueba}{\textbf{Prueba: }}
\newcommand{\proposicion}{\textbf{Proposición: }}
\newcommand{\corolario}{\textbf{Corolario: }}
\newcommand{\grad}{\text{grad}}

% Subtitles
\newcommand{\subtitle}[1]{%
  \posttitle{%
    \par\end{center}
    \begin{center}\large#1\end{center}
    \vskip0.5em}%
}

% Definition of sections and their numbering 

\makeatletter
\def\@seccntformat#1{%
  \expandafter\ifx\csname c@#1\endcsname\c@section\else
  \csname the#1\endcsname\quad
  \fi}
\makeatother

%opening
{\title{{\huge Fundamentos de la Mecánica} \\
\vspace{.2cm}
\large Una traducción no oficial del libro \emph{Foundations of Mechanics} (2da edición) 
de Ralph Abraham y Jerrold Marsden}
\author{Favio Vázquez\thanks{Correo: favio.vazquezp@gmail.com}}\affil{Instituto de Ciencias Nucleares. Universidad Nacional Autónoma de México.}
\date{}

\begin{document}

\makeatletter
\def\@maketitle{%
  \newpage
  \null
  \vskip 2em%
  \begin{center}%
  \let \footnote \thanks
    {\Large\bfseries \@title \par}%
    \vskip 1.5em%
    {\normalsize
      \lineskip .5em%
      \begin{tabular}[t]{c}%
        \@author
      \end{tabular}\par}%
    \vskip 1em%
    {\normalsize \@date}%
  \end{center}%
  \par
  \vskip 1.5em}
\makeatother

\maketitle

\section*{Vista Previa}

\vspace{3cm}

Para motivar la introducción a la geometría simpléctica en la mecánica, necesitamos 
considerar brevemente las ecuaciones de Hamilton. El punto de partida es la segunda 
ley de Newton, que establece que una partícula de masa $m > 0$ que se mueve en un 
potencial $V(q)$, $q=(q^1,q^2,q^3) \in \mathbf{R}^3$, se mueve a lo largo de una 
curva $q(t)$ en $\mathbf{R}^3$ de manera tal que $m\ddot{q} = - \grad{V(q)}$. Si 
introducimos el impulso $p_i = m\dot{q}^i$ y la energía $H(q,p) = (1/2m)||p||^2 + V(q)$,
entonces la ecuación de Newton es equivalente a las ecuaciones de Hamilton:

\begin{align*}
 \begin{cases}
  \dot{q}^i = \partial H/\partial p_i \\
  \dot{p}_i = - \partial H/\partial q^i,          & \text{i=1,2,3}
\end{cases}
\end{align*}

Luego uno procede a estudiar este sistema de ecuaciones de primer orden para un $H(q,p)$ 
general. Para hacer esto, introducimos la matriz $J = \begin{pmatrix} 0 & I \\ -I & 0\end{pmatrix}$, 
donde $I$ es la identidad $3\times 3$, y notamos que las ecuaciones se convierten en 
$\dot{\xi} = J\cdot \grad H(\xi)$, donde $\xi = (q,p)$. (En notación compleja, 
haciendo $z=q+ip$, éstas pueden escribirse como $\dot{z} = -2i\partial G/\partial\overline{z}$.)

\vspace{.3cm}

Hagamos $X_H = J\cdot \grad H$. Entonces $\xi(t)$ satisface las ecuaciones de Hamilton
sí y solo sí $\xi(t)$ es una curva integral de $X_H$, es decir, $\dot{\xi}(t) = X_H(\xi(t))$. La 
relación entre $X_H$ puede ser rescrita como sigue: introducimos la forma bilineal anti-simétrica 
$\omega$ sobre $\mathbf{R}^3\times\mathbf{R}^3$ definida por 

\begin{align*}
 \omega(v_1,v_2) = v_1 \cdot J \cdot v_2 \\
 v_1, v_2 \in \mathbf{R}^3\times\mathbf{R}^3 \\
 v_1 = (x_1,y_1) \\
 v_2 = (x_2,y_2)
\end{align*}

[En notación compleja sobre $C^3 \cong \mathbf{R}^3\times\mathbf{R}^3$, 
$\omega(v_1,v2) = - Im\langle v_1, v_2 \rangle$, donde $v_1 = x_1 + i y_1$, 
$v_2 = x_2 + i y_2$, y $\langle \hspace{.1cm}, \hspace{.1cm} \rangle$ es el producto interno Hermitiano.]

\vspace{.3cm}

Entonces tenemos, para todo $\xi \in \mathbf{R}^3\times\mathbf{R}^3$ y $v \in \mathbf{R}^3\times\mathbf{R}^3$,

\begin{equation*}
 \omega (X_H(\xi), v) = dH(\xi)\cdot v
\end{equation*}

donde $dH(q,p) = (\partial H/\partial q^i, \partial H/\partial p^i)$, un vector 
columna en $\mathbf{R}^3\times\mathbf{R}^3$, como es fácilmente comprobable. Uno llama 
$\omega$ la \emph{forma simpléctica} sobre $\mathbf{R}^3\times\mathbf{R}^3$, y $X_H$ 
el \emph{campo vectorial hamiltoniano} con energía $H$.

\vspace{.3cm}

Supongamos que hacemos el cambio de coordenadas $\eta = f(\xi)$, donde $f: \mathbf{R}^3\times\mathbf{R}^3 
\rightarrow \mathbf{R}^3\times\mathbf{R}^3$ es suave. Si $\xi(t)$ satisface las ecuaciones 
de Hamilton, las ecuaciones satisfechas por $\eta(t) = f(\xi(t))$ son 
$\dot{\eta} = A\dot{\xi} = AJ\grad_\xi H(\xi) = AJA^*\grad_\eta H(\xi(\eta)$, donde 
$(A)^{ij} = (\partial \eta^i/\partial \xi^j)$ es el jacobiano de $f$, y $A^*$ es la 
transpuesta de $A$. Las ecuaciones para $\eta$ serán hamiltonianas con energía 
$K(\eta) = H(\xi(\eta))$ sí y solo sí $AJA^* = J$. Una transformación que satisface 
esta condición es llamada \emph{canónica} o \emph{simpléctica}, (o un simplectomorfismo).
En términos de la forma simpléctica $\omega$, esta condición, denotada por $f^*\omega = \omega$, 
dice que la transformación $f$ no cambia a $\omega$. 

\vspace{.3cm}

El espacio $\mathbf{R}^3\times\mathbf{R}^3$ de los $\xi$'s es llamado el \emph{espacio de 
fases}. Para un sistema de $N$ partículas utilizaríamos $\mathbf{R}^3N\times\mathbf{R}^3N$. 

\vspace{.3cm}

Para muchos sistemas físicos fundamentales, el espacio de fases es una variedad más bien 
que un espacio euclidiano. Por ejemplo, las variedades surgen frecuentemente cuando 
las constricciones están presentes. Por ejemplo, el espacio de fases para el movimiento 
de un cuerpo rígido es el haz tangente del grupo $SO(3)$ de las matrices ortogonales 
$3 \times 3$ con determinante $+1$. No solo las variedades son importantes en estos
ejemplos, sino que también su terminología y notación llevan a un entendimiento más 
claro de la estructura básica de la mecánica.

\newpage

\resizebox{.5\linewidth}{!}{\itshape \textbf{Parte 1}}

\vspace{.3cm}

\noindent\rule[0.5ex]{.55\linewidth}{1pt}

\resizebox{.5\linewidth}{!}{\itshape \textbf{Preliminares}}

\noindent\rule[0.5ex]{.55\linewidth}{1pt}

\vspace{4cm}

Las herramientas básicas necesarias para nuestro estudio de la mecánica se desarrollan
aquí. Herramientas más especializadas son desarrolladas luego cuando se necesiten. 
Aquellos con el entrenamiento matemático suficiente pueden por supuesto saltarse
esta parte luego de familiarizarse con nuestra notación. Obviamente uno no puede 
esperar dominar todos los preliminares si uno está comenzando desde cero, sin un 
esfuerzo masivo. Por lo tanto, parece sabio primero ir por esta parte rápidamente y 
luego, comenzando con la Parte 2, volver cuando surja la ocasión para un segundo 
estudio más serio.



\newpage 

\section{{\huge{Capítulo 1}}}

\noindent\rule[0.5ex]{\linewidth}{1pt}

\vspace{.3cm}

{\Large \textbf{Teoría diferencial}}

\vspace{4cm}

Las categorías de las variedades diferenciales y de los haces vectoriales proveen un 
contexto útil para las matemáticas necesitadas en la mecánica, especialmente los 
nuevos resultados topológicos y cualitativos. Este capítulo desarrolla estas categorías.
Las herramientas para este desarrollo --- la topología y el cálculo en espacio lineales ---
se estudian primero.

\subsection{Topología}

Uno de las mayores dificultades que este libro presente a los no matemáticos es la 
dependencia en la topología general. Aunque referencias excelentes están disponibles, 
los tópicos requeridos no pueden encontrarse todos en un solo texto, y la variación 
de notaciones y orden en los diferentes libros presentan un reto difícil para el 
lector inexperto. Ensamblamos aquí para referencia los tópicos necesarios, en una 
notación consistente usada a lo largo de este libro. Un número de pruebas más 
técnicas que no son relevantes para nosotros son omitidas y el lector es referido 
a un texto estándar. 

\vspace{.3cm}

Esta sección no pretende reemplazar un curso completo de topología y el lector no debe 
esperar dominarla en una primera lectura.

\vspace{.3cm}

Se asume que el lector está familiarizado con las notaciones usuales de teoría de 
conjuntos como $\in$, $\cup$, $\cap$ y con el concepto de un mapeo. Si 
$A$ y $B$ son conjuntos y $f:A \rightarrow B$ es un mapeo, usualmente 
escribimos $a \mapsto f(a)$ el efecto de el mapeo sobre el elemento 
$a \in A$; ``ssí'' representa ``sí y solo sí'' (= ``si'' en definiciones).

\subsubsection{Definiciones} \emph{Un \textbf{espacio topológico} es un 
conjunto $S$ junto con una colección $\mathcal{O}$ de subconjuntos llamados 
\textbf{conjuntos abiertos} tales que}

\begin{enumerate}[label=(T{\arabic*})]
 \item $\varnothing \in \mathcal{O} \mathcal{}$ y $S \in \mathcal{O}$;
 \item \emph{Si $U_1,U_2 \in \mathcal{O}$, entonces $U_1 \cap U_2 \in \mathcal{O}$};
 \item \emph{La unión de cualquier colección de conjuntos abiertos es abierta.}
\end{enumerate}

\emph{Para tal espacio topológico los \textbf{conjuntos cerrados} son 
los elementos de}

$$
\Gamma = \{A | \mathcal{C} A \in \mathcal{O} \}
$$

\emph{Donde $\mathcal{C}$ denota el complemento, $\mathcal{C}A = S {\char`\\} A = \{s \in S | s \notin A\}$.
(Los conjuntos cerrados entonces obedecen las reglas duales a las de los conjuntos abiertos.}

\vspace{.3cm}

\newpage

\section{{\huge{Capítulo 3}}}

\noindent\rule[0.5ex]{\linewidth}{1pt}

\vspace{.3cm}

{\Large \textbf{Sistemas hamiltonianos y lagrangianos}}

\vspace{4cm}

Este capítulo comienza nuestro estudio de la mecánica hamiltoniana. La 
estructura básica de la teoría clásica será dada en el contexto de 
variedades. Sugerimos al lector tener un buen texto disponible para 
comparación y una visión adicional, como Whittaker [1959] o Goldstein [1950]. 
Una re-lectura de la Vista Previa en este momento puede ayudar a motivar 
lo que sigue. El tratamiento podrá parecer innecesariamente abstracta, 
pero es de un beneficio final para un entendimiento minucioso de un 
análisis riguroso de las aplicaciones en posteriores capítulos.

\subsection{Álgebra simpléctica}

Las variedades simplécticas constituyen el escenario para la mecánica 
hamiltoniana. Esta sección considera el caso lineal en la preparación 
para la siguiente sección.

\subsubsection{Definición}

\emph{Sea \textbf{E} un espacio vectorial real de dimensión finita y 
$\mathbf{\omega} \in L^2(\mathbf{E},\mathbf{R})$ una forma bilineal
sobre $\mathbf{E}$, por lo que $\mathbf{\omega}: \mathbf{E}\times \mathbf{E} 
\rightarrow \mathbf{R}$. Decimos que $\mathbf{\omega}$ es no degenerada 
si}

$$
\mathbf{\omega}(\mathbf{e}_1, \mathbf{e}_2) = 0 \quad \text{\emph{para toda}}
\quad \mathbf{e}_2 \in \mathbf{E} \quad \text{\emph{implica}} \quad 
\mathbf{e}_1 = 0
$$

\emph{Hay algunas maneras equivalentes de expresar la no degeneración. Para 
dar estas, necesitamos un poco de notación:}

\begin{enumerate}[label=({\alph*})]
 \item \emph{Si $\hat{\mathbf{e}} = (\mathbf{e}_i)$ es una base 
 (ordenada) de $\mathbf{E}$ y $(\mathbf{\alpha}^i)$ es la base dual, 
 $\omega_{ij} = \mathbf{\omega}(\mathbf{e}_i,\mathbf{e}_j)$ es la 
 \textbf{matrix} de $\mathbf{\omega}$; es denotada $[\mathbf{\omega}]_{\hat{\mathbf{e}}}$. 
 Entonces es fácil ver que }
 
 $$
 \mathbf{\omega} = \omega_{ij} \mathbf{\alpha}^i \otimes \mathbf{\alpha^j} \quad \text{\emph{(suma entendida)}}
 $$
 
 \item \emph{La \textbf{transpuesta} $\mathbf{\omega}^i$ de $\mathbf{\omega}$ está dada 
 por }
 
 $$
 \mathbf{\omega}^t(\mathbf{e}_1,\mathbf{e}_2) = \mathbf{\omega}(\mathbf{e}_2,\mathbf{e}_1);
 $$
 
 \emph{$\mathbf{\omega}$ es \textbf{simétrica} si $\mathbf{\omega}^t = \mathbf{\omega}$ 
 y \textbf{anti-simétrica} si $\mathbf{\omega}^t = - \mathbf{\omega}$.}
 
 \item \emph{El mapeo lineal $\mathbf{\omega}^b: \mathbf{E} \rightarrow \mathbf{E}^*$ 
 está definido por}
 
 $$
 \mathbf{\omega}^b(\mathbf{e})\cdot\mathbf{e}' = \mathbf{\omega}(\mathbf{e},\mathbf{e}')
 $$
 
 \emph{notemos que la matriz $\mathbf{\omega}^b$ relativa a las bases $\mathbf{e}_i$ y 
 $\mathbf{\alpha}^j$ es exactamente $\omega_{ij}$, es decir,} 
 
 $$
 \mathbf{\omega}^b(\mathbf{e}_i) = \omega_{ij}\mathbf{\alpha}^j
 $$
 
 \emph{el rango de $\mathbf{\omega}$ es el rango de la matriz $\omega_{ij}$, es decir, 
 la dimensión de $\mathbf{\omega}^b(\mathbf{E})$.}
\end{enumerate}

Si uno usa (i) el hecho de que en dimensiones finitas un mapeo lineal uno-a-uno 
entre espacio de la misma dimensión es un isomorfismo y (ii) la observación de 
que la definición de no degeneración es precisamente que el mapeo lineal $\mathbf{\omega}^b$ 
es uno-a-uno; es decir, tiene un núcleo trivial, es fácil ver que lo siguiente 
es equivalente:

\begin{enumerate}[label=({\roman*})]
 \item $\mathbf{\omega}$ es no degenerado;
 \item $\mathbf{\omega}^t$ es no degenerado;
 \item la matriz de $\mathbf{\omega}$ es no singular;
 \item $\mathbf{\omega}^t:\mathbf{E} \rightarrow \mathbf{E}^*$ es un isomorfismo.
\end{enumerate}


\end{document}